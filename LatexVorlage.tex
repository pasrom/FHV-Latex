%
% 1: documentation, e.g.: a documentation for a project done in a course
% 2: thesis, e.g.: a master thesis
% 3: summary
% 5: presentation
% 9: paper
%
\ifx\FHVmode\undefined
	\def\FHVmode{2}
\fi
%
% 1: Presentation only
% 2: Presentation only notes
% 3: Presentation all (left page is the presentation, right the notes)
%
\ifx\PresentationMode\undefined
	\def\PresentationMode{3}
\fi
%
\newcommand{\version}{v0.0}
%
%
% only \def and \newcommand{cmd}{def} allowed here!
% because no packages are loaded here
%
\def\pathName{Programm}
\def\ValueBindingOffset{0mm} 
\def\debug{false}
%
% document types
%
\def\documentation{1}
\def\thesis{2}
\def\summary{3}
\def\presentations{5}
\def\paper{9}
%
% presentation build types
%
\def\PresentationOnly{1}
\def\PresentationNotes{2}
\def\PresentationAll{3}
%
% language types
%
\def\ngerman{ngerman}
%
% define if you want to show notes
% disable: notes not showed
% draft:   notes showed
%
\def\notesFHV{draft}
%
% Uncomment this, if you want another styled title page
%
%\def\FHVtitlePage{fhv}
%
% english or ngerman
%
\ifx\newLanguage\undefined
	\def\newLanguage{ngerman} % build twice!!!
%	\def\newLanguage{english} % build twice!!!
\fi
%
% Defines to make live easier...
% ATTENTION: do not forget to put a \xspace after the command!
%
\def\pfeil{\ensuremath{\rightarrow}\xspace}
\def\authorName{Name\xspace}
\def\authorSurname{Surname\xspace}
% uncomment \authorTitleBefore & \authorTitleAfter if not used
\def\authorTitleBefore{Title B\xspace}
\def\authorTitleAfter{Title A\xspace}
\def\supervisorName{Supervisor Name\xspace}
\def\supervisorSurname{Supervisor Surname\xspace}
% uncomment \supervisorTitleBefore & \supervisorTitleAfter if not used
\def\supervisorTitleBefore{Title B\xspace}
\def\supervisorTitleAfter{Title A\xspace}
%
% uncomment this if you like a title, e.g. for the mechatronik plattform
%
%\def\paperName{KONFERENZ DER MECHATRONIK-PLATTFORM: Smart Factory\\
%							FH Technikum Wien,  24. November 2016\xspace}
%
% Define if the author is male (m) or female (w)
%
\def\wOrM{m}
%
% pdf settings
%
\makeatletter
	\immediate\write18{git log -1 --pretty=format:"\@backslashchar gdef\@backslashchar GITHash{\@percentchar h}\@percentchar n\@backslashchar gdef\@backslashchar GITDate{\@percentchar ad}" --date=short > build/git-info.txt}
\makeatother
\input{build/git-info.txt}
%
\def\pdfSettings{%
	%
	\def\tmpPdfSubject{\ifdefined\GITHash%
		git-short:~\GITHash%
	\fi\ifdefined\GITDate%
		~git-date:~\GITDate%
	\fi}
	\hypersetup{pdftitle={\getTitle},%
		pdfauthor={\authorSurname~\authorName},%
		pdfsubject={\tmpPdfSubject},%
		pdfkeywords={Fachhochschule Vorarlberg : \getThesistype~: \studyType~: \supervisorSurname, \supervisorName},%
		pdflang={de},%
		unicode=true,%
		%
		% hyperrefsetup: colour of hyperlinks instead of boxes 
		%
		colorlinks   = false, % true: colour text, false: colour box
		urlcolor     = blue,  % external colour links
		linkcolor    = black, % internal links colour
		citecolor    = black, % colour of citations
		}%
}
%
\pdfminorversion=7 % use newer pdf version
\pdfobjcompresslevel=2
\pdfsuppresswarningpagegroup=1
%
% standard definiton for biblatex
%
\def\biblatexOptions{%
	backend=biber,%
	style=authoryear,%
	citestyle=authoryear,%
	dashed=false,%
	backref=true,%
}
%
% standard definiton for the document class
%
\def\komaScriptClass{scrreprt}%
\def\documentclassOptions{%
	a4paper,%
	oneside,%
	fontsize=11pt,%
	headsepline,%
	headings=big,%
	draft=\debug,%
	numbers=noenddot,%
	listof=totoc,%
}
%
% paper definitons
%
\if\paper\FHVmode
	\def\komaScriptClass{scrartcl}%
	\def\documentclassOptions{%
		a4paper,%
		oneside,%
		fontsize=9pt,%
%		draft=\debug,%
		numbers=enddot,%
		twocolumn,
		bibliography=totocnumbered,%
	}%
	\def\biblatexOptions{%
		backend=biber,%
		style=authoryear,%
		citestyle=authoryear,%
		dashed=false,%
	}
\fi
%
% summary definitions
%
\if\summary\FHVmode
\def\komaScriptClass{scrreprt}%
\def\documentclassOptions{%
	a4paper,%
	oneside,%
	fontsize=11pt,%
	headings=big,%
	draft=\debug,%
	numbers=noenddot,%
	listof=totoc,%
	numbers=noenddot,%
}
\fi%
%
% presentation definitions
%
\if\presentations\FHVmode
\def\komaScriptClass{beamer}%
\def\documentclassOptions{%
	18pt,%
	xcolor=dvipsnames,%
	hyperref={breaklinks=true},%
	xcolor=table,%
}%
\fi
%
% Color for tables
% 
\def\farbeTabA{C0C0C0}
\def\farbeTabB{EFEFEF}
%
% Commands to make live easier...
%
\newcommand{\mtnote}[1]{\textsuperscript{\TPTtagStyle{#1}}}
%
\def\SymbReg{\textsuperscript{\textregistered}}
\newcommand{\MATLAB}{\textsc{Matlab\small\SymbReg}\xspace}
%
% do the caption with a source
%
\newcommand{\unterschrift}[3]%
{%
	\ifthenelse{\equal{\getLanguage}{english}}% english or german
	{\def\quelle{Source}}%
	{\def\quelle{Quelle}}%
	\def\source{\ifthenelse{\equal{#2}{}}{}{\\\quelle: #2}}%
	% if no source is given, don't use this
	\ifthenelse{\equal{#3}{no}}% add to the table of contents? Standard is on
	{\caption[]{#1\source}}%
	{\caption[#1]{#1\source}}%
}%
%
\newcommand{\FHVcheckbox}[1]%
{%
	\def\checked{1}
	\def\unchecked{0}
	\if#1\checked
		\makebox[0pt][l]{$\square$}\raisebox{.15ex}{\hspace{0.1em}\(\checkmark\)}
	\fi
	\if#1\unchecked
		\makebox[0pt][l]{$\square$}\raisebox{.15ex}{\hspace{0.1em}}\hspace{1em}
	\fi
}
%

%
\documentclass[\documentclassOptions]{\komaScriptClass}
%
\usepackage{./sty/fhv}
%
\makeglossaries
%
% General settings for title...
%
\setLanguage{\newLanguage}
\setTitle{Title}
\setThesistype{Master Thesis}
\setAuthor{\authorSurname\authorName}
\setAuthorId{MATRIKELNUMMER}
\setStudyprogram{Master's in Mechatronics}
\setSupervisor{\supervisorSurname\supervisorName}
\setSupervisorCompany{Title B SupervisorCompanyName, Title A}
\setSubtitle{Subtitle}
\setSubject{Subject}
\setDegree{Master of Science in Engineering, MSc}
\setCompany{Company Name GmbH}
%
% hyperrefsetup: colour of hyperlinks instead of boxes 
%
\hypersetup{
	colorlinks   = false,  % true: colour text, false: colour box
	urlcolor     = blue,  % external colour links
	linkcolor    = black, % internal links colour
	citecolor    = black  % colour of citations
}
%
% PDF settings
%
\pdfSettings
%
\allowdisplaybreaks
%
\begin{document}
	%
	% Select the language defined in \newLanguage
	%
	\ifx\newLanguage\ngerman
		\selectlanguage{ngerman}
	\else
		\selectlanguage{english}
	\fi %
	%
	\if\FHVmode\paper
		\SetAlgorithmName{Algorithmus}{Alg.}
	\makeatletter
		\crefname{equation}{Gl.}{Gln.}
	\makeatother
	\fi
	% Import the acronyms
	%
	\newacronym{msrp}{MSRP}{Multiple Stream Registration Protocol}
\newacronym{profinet}{PROFINET}{Process Field Network}
\newacronym{bluecom}{BCP}{bluecom Protokoll}
\newacronym{iperf}{iPerf}{The network bandwidth measurement tool}
\newacronym{bb}{BB}{BeagleBone Black}
\newacronym{fifo}{FIFO}{First-In, First-Out}
\newacronym{nettst}{NETTST}{Netzwerk Test}
\newacronym{poc}{PoC}{Proof of Concept}
\newacronym{ptp}{PTP}{Precision Time Protocol}
\newacronym{tsn}{TSN}{Time Sensitive Networking}
\newacronym{cbs}{CBS}{Credit Based Traffic Shaping}
\newacronym{srp}{SRP}{Stream Reservation Protocol}
\newacronym{lldp}{LLDP}{Link Layer Discovery Protocol}
\newacronym{iot}{IoT}{Internet of Things}
\newacronym{irt}{PROFINET IRT}{PROFINET Isochronous Real-Time}
\newacronym{ethercat}{EtherCAT}{Ethernet for Control Automation Technology}
\newacronym{bmca}{BMCA}{Best Master Clock Algorithm}
\newacronym{tas}{TAS}{Time Aware Shaper}
\newacronym{tbs}{TBS}{Credit Based Shaper}
\newacronym{isis}{IS-IS}{Intermediate System to Intermediate System}
\newacronym{vlan}{VLAN}{Virtual Local Area Network}
\newacronym{qos}{QoS}{Quality of Service}
\newacronym{rt}{RT}{Real Time}
\newacronym{tcp}{TCP}{Transmission Control Protocol}
\newacronym{fhv}{FHV}{Fachhochschule Vorarlberg}
\newacronym{ua}{UA}{Unified Architecture}
\newacronym{opc}{OPC}{Open Platform Communications}
\newacronym{ip}{IP}{Internet Protocol}
\newacronym{soc}{SoC}{System-on-a-Chip}
\newacronym{cpu}{CPU}{Central Processing Unit}
\newacronym{arm}{ARM}{Advanced \gls{risc} Machines}
\newacronym{risc}{RISC}{Reduced Instruction Set Computer}
\newacronym{nic}{NIC}{Network Interface Card}
\newacronym{mac}{MAC}{Media-Access-Control}
\newacronym{arp}{ARP}{Address Resolution Protocol}
\newacronym{mtu}{MTU}{Maximum Transmission Unit}
\newacronym{sw}{SW}{Software}
\newacronym{raspy}{Raspberry Pi}{Raspberry Pyhton Interpreter}
\newacronym{udp}{UDP}{User Datagram Protocol}
\newacronym{gui}{GUI}{Graphical User Interface}
\newacronym{rstp}{RSTP}{Rapid Spanning Tree Protocol}
\newacronym{sof}{SoF}{Start of Frame}
\newacronym{pbid}{TPID}{Tag Protocol Identifier}
\newacronym{tci}{TCI}{Tag Control Information}
\newacronym{pcp}{PCP}{Priority Code Point}
\newacronym{dei}{DEI}{Drop Eligible Indicator}
\newacronym{vid}{VID}{\glsentryshort{vlan}-Identifier}
\newacronym{sfd}{SFD}{Start of frame delimiter}
\newacronym{crc}{CRC}{Cyclic Redundancy Check}
\newacronym{fcs}{FCS}{Frame Check Sequence}
\newacronym{tpid}{TPID}{Tag Protocol Identifier}
\newacronym{p2p}{P2P}{Peer-to-Peer}
\newacronym{tsu}{TSU}{Zeitstempeleinheit}
\newacronym{spf}{SPF}{Shortest Path First}
\newacronym{lsdb}{LSDB}{Link State Database}
\newacronym{fpu}{FPU}{Fließkommaeinheit}
\newacronym{ftp}{FTP}{File Transfer Protocol}
\newacronym{icmp}{ICMP}{Internet Control Message Protocol}
\newacronym{cgi}{CGI}{Common Gateway Interface}
\newacronym{hdmi}{HDMI}{High Definition Multimedia Interface}
\newacronym{fpga}{FPGA}{Field Programmable Gate Array}
\newacronym{asic}{ASIC}{Application-Specific Integrated Circuit}
\newacronym{par}{PAR}{Project Authorization Request}
\newacronym{revcom}{RevCom}{Review Committee}
\newacronym{gps}{GPS}{Global Positioning System}
\newacronym{isoosi}{ISO/OSI}{ International Organization for Standardization / Open Systems Interconnection Model}
\newacronym{dns}{DNS}{Domain Name System}
\newacronym{dhcp}{DHCP}{Dynamic Host Configuration Protocol}
\newacronym{rip}{RIP}{Address Resolution Protocol}
\newacronym{ttl}{TTL}{Time to live}
\newacronym{wpad}{WPAD}{Web Proxy Auto-Discovery Protocol}
\newacronym{rpc}{RPC}{Remote Procedure Call}
\newacronym{rfc}{RFC}{Remote Function Call}
\newacronym{wpa}{WPA}{Wi-Fi Protected Access}
\newacronym{http}{HTTP}{Hypertext Transfer Protocol}
\newacronym{https}{HTTPS}{Hypertext Transfer Protocol Secure}
\newacronym{smtp}{SMTP}{Simple Mail Transfer Protocol}
\newacronym{pop}{POP}{Post Office Protocol }
\newacronym{telnet}{Telnet}{Teletype Network}
\newacronym{rtc}{RTC}{Real Time Clock}
\newacronym{twi}{TWI}{Two Wire Interface}
\newacronym{wdt}{WDT}{Watch Dog Timer}
\newacronym{wic}{WIC}{Wake Up Interrupt Controller}
\newacronym{isr}{ISR}{Interrupt Service Routine}
\newacronym{prr}{PRR}{Power Reduction Registers}
\newacronym{bod}{BOD}{Brown Out Detector}
\newacronym{ocd}{OCD}{On-chip Debugging}
\newacronym{ram}{RAM}{Read Access Memory}
\newacronym{cc}{CC}{Compare Match}
\newacronym{ov}{OV}{Overflow}
\newacronym{un}{UN}{Underflow}
\newacronym{tc}{TC}{Terminal Count}
\newacronym{pwm}{PWM}{Pulse Width Modulation}
\newacronym{lfsr}{LFSR}{linear feedback shift register}
\newacronym{bist}{BIST}{Built-in self-test}
\newacronym{saf}{SAF}{Stuck-At Fault}
\newacronym{tf}{TF}{Transition Fault}
\newacronym{cf}{CF}{Coupling Fault}
\newacronym{npsf}{NPSF}{Neighborhood Pattern Sensitive Fault}
\newacronym{af}{AF}{Address decoding fault}
\newacronym{ic}{IC}{integrated circuit}
\newacronym{rom}{ROM}{read only memory}
\newacronym{dma}{DMA}{direct memory access}
\newacronym{dsp}{DSP}{digital signal processor}
\newacronym{sram}{SRAM}{Static random-access memory}
\newacronym{can}{CAN}{Controller Area Network}
\newacronym{sdo}{SDO}{Service Data Object}
\newacronym{pdo}{PDO}{Process Data Object}
\newacronym{nmt}{NMT}{Network Management }
\newacronym{time}{TIME}{Time Stamp Object}
\newacronym{ucmm}{UCMM}{Unconnected Message Manager}
\newacronym{fms}{FMS}{Field Message Service}
\newacronym{dp}{DP}{Decentralized Peripherals}
\newacronym{fdl}{FDL}{Fieldbus Data Link}
\newacronym{mbp}{MBP}{Manchester Coded, Bus Powered}
\newacronym{fisco}{FISCO}{Fieldbus Intrinsically Safe Concept}
\newacronym{pa}{PA}{Prozess-Automation}
\newacronym{dpm}{DPM}{DP-Master-Klasse}
\newacronym{dpm1}{DPM1}{DP-Master-Klasse 1}
\newacronym{dpm2}{DPM2}{DP-Master-Klasse 2}
\newacronym{sps}{SPS}{speicherprogrammierbare Steuerungen}
\newacronym{hmi}{HMI}{Human-Machine-Interfaces}
\newacronym{cba}{CBA}{Component Based Automation}
\newacronym{ar}{AR}{Application Relation}
\newacronym{cr}{CR}{Communication Relations}
\newacronym{wlan}{WLAN}{Wireless Local Area Network}
\newacronym{rtr}{RTR}{Telekom-Control-Kommission}
\newacronym{ism}{ISM}{Industrial Scientic Medical}
\newacronym{ble}{BLE}{Bluetooth Low Energy}
\newacronym{fhss}{FHSS}{Frequency hopping spread spectrum}
\newacronym{dsss}{DSSS}{Direct sequence spread spectrum}
\newacronym{dbd}{dBd}{Dezibel Dipol}
\newacronym{dbi}{dBi}{Dezibel Isotropic}
\newacronym{los}{LOS}{Line of Sight}
\newacronym{ber}{BER}{bit error rate}
\newacronym{bt}{BT}{Bluetooth}
\newacronym{tdm}{TDM}{Time Division Multiplexing}
\newacronym{fdm}{FDM}{Frequency Division Multiplexing}
\newacronym{cdm}{CDM}{Code Division Multiplexing}
\newacronym{pam}{PAM}{Pulse Amplitude Modulated}
\newacronym{am}{AM}{Analog Modulation}
\newacronym{fm}{FM}{Frequency Modulation}
\newacronym{ask}{ASK}{Amplitude Shift Keying}
\newacronym{fsk}{FSK}{Frequency Shift Keying}
\newacronym{psk}{PSK}{Phase Shift Keying}
\newacronym{bpsk}{BPSK}{Binary Phase Shift Keying}
\newacronym{qpsk}{QPSK}{Quadrature Phase Shift Keying}
\newacronym{qam}{QAM}{Quadrature Amplitude Modulation}
\newacronym{sdm}{SDM}{Space Division Multiplexing}
\newacronym{llc}{LLC}{Logical Link Control}
\newacronym{csma}{CSMA}{Carrier Sense Multiple Access}
\newacronym{cd}{CD}{Collision Detection}
\newacronym{cdma}{CDMA}{Code Division Multiple Access}
\newacronym{sdma}{SDMA}{Space Division Multiple Access}
\newacronym{fdma}{FDMA}{Frequency Division Multiple Access}
\newacronym{tdma}{TDMA}{Time Division Multiple Access}
\newacronym{ca}{CA}{Collision Avoidence}
\newacronym{rts}{RTS}{Request to Send}
\newacronym{cts}{CTS}{Clear to Send}
\newacronym{gatt}{GATT}{Generic Attribute}
\newacronym{gap}{GAP}{Generic Access Profile}
\newacronym{l2cap}{L2CAP}{Logical Link Control and Adaption Protocol}
\newacronym{hci}{HCI}{Host Controller Interface}
\newacronym{mitm}{MITM}{Man in the Middle}
\newacronym{ack}{ACK}{Acknowledgment}
\newacronym{ll}{LL}{Link Layer}
\newacronym{gfsk}{GFSK}{Gaussian Frequency Shift Keying}
\newacronym{att}{ATT}{Attribute Protocol}
\newacronym{sm}{SM}{Security Manager}
\newacronym{poe}{PoE}{Power over Ethernet}
\newacronym{cs}{CS}{Carrierer Sense}
\newacronym{jtag}{JTAG}{Joint Test Action Group}
\newacronym{ppb}{PPB}{Private Peripheral Bus}
\newacronym{irq}{IRQ}{lnterrupt Request}
\newacronym{uma}{UMA}{Uniform memory access}
\newacronym{numa}{NUMA}{Nonuniform memory access}
\newacronym{an}{AN}{artificial Neuron}
\newacronym{gd}{GD}{gradient descent}
\newacronym{lti}{LTI}{Linear Time Invariant}
\newacronym{mimo}{MIMO}{multiple Input muliple Output}
\newacronym{ode}{ODE}{ordinary differential equation}
\newacronym{fft}{FFT}{Fast Fourier Transformation}
\newacronym{adc}{ADC}{Analog Digital Wandler}
\newacronym{dac}{DAC}{Digital Analog Wandler}
\newacronym{snr}{SNR}{Signal Noice Ratio}
\newacronym{sandh}{S\&H}{Sample und Hold}
\newacronym{dit}{DIT}{Discrete in Time}
\newacronym{dis}{DIS}{Discrete in Space}
\newacronym{stft}{STFT}{Short Term Fourier Transform}
\newacronym{ft}{FT}{Fourier Transformation}
\newacronym{dft}{DFT}{digitale Fourier Transformation}
\newacronym{aaf}{AAF}{Anti Aliasing Filter}
\newacronym{fir}{FIR}{Finite Impuls Response}
\newacronym{iir}{IIR}{Infinite Impuls Response}
\newacronym{ghpf}{GHPF}{Gaussian High Pass Filter}
\newacronym{glpf}{GLPF}{Gaussian Low Pass Filter}
\newacronym{bhpf}{BHPF}{Butterworth High Pass Filter}
\newacronym{blpf}{BLPF}{Butterworth Low Pass Filter}
\newacronym{ilpf}{BLPF}{Ideal Low Pass Filter}
\newacronym{ihpf}{IBPF}{Ideal High Pass Filter}
\newacronym{dtmf}{DTMF}{Dual-Tone multi-Frequency}
\newacronym{wt}{WT}{Wavelet Transformation}
\newacronym{wft}{WFT}{Windowed Fourier Transformation}
\newacronym{dwt}{DWT}{Discrete Wavelet Transformation}
\newacronym{wpt}{WPT}{Wavelet Packet Transform}
\newacronym{cwt}{CWT}{Continuous Wavelet Transformation}
\newacronym{pir}{PIR}{Pyroelectric Infrared}
\newacronym{hmm}{HMM}{Hidden Markov Models}
\newacronym{cgmm}{CGMM}{Conditional Gaussian Mixture Models}
\newacronym{pca}{PCA}{Principal Component Analysis}
\newacronym{svm}{SVM}{Support Vector Machine}
\newacronym{fce}{FCE}{Fuzzy Comprehensive Evaluation}
\newacronym{ht}{HT}{Hilbert Transformation}
\newacronym{ga}{GA}{Genetische Algorithmus}
\newacronym{rms}{RMS}{Root Mean Square}
\newacronym{knn}{KNN}{k-Nearest-Neighbor-Algorithmus}
\newacronym{fr}{FR}{funktionale Anforderung}
\newacronym{nfr}{NFR}{nicht funktionale Anforderung}
\newacronym{cb}{CB}{Conveyor Belt}
\newacronym{uml}{UML}{Unified Modeling Language}
\newacronym{ros}{ROS}{Roboter Operating System}
\newacronym{slam}{SLAM}{Simultaneous Localization and Mapping}
\newacronym{svo}{SVO}{Fast Semi-Direct Visual Odometry}
\newacronym{imu}{IMU}{inertial measurement unit}
\newacronym{pcl}{PCL}{Pointcloud}
\newacronym{dlt}{DLT}{Direct Linear Transformation}
\newacronym{svd}{SVD}{Singular Value Decomposition}
\newacronym{pk}{PK}{Punktkorrespondenz}
\newacronym{ri}{RI}{Redundancy Interconnection}
\newacronym{snmp}{SNMP}{Simple Network Management Protocol}
\newacronym{uni}{UNI}{User / Network Interface}
\newacronym{tlv}{TLV}{Type Length Value}
\newacronym{xml}{XML}{Extensible Markup Language}
\newacronym{json}{JSON}{JavaScript Object Notation}
\newacronym{cnc}{CNC}{Centralized Network Configuration}
\newacronym{cuc}{CUC}{Centralized User Configuration}
\newacronym{scada}{SCADA}{Supervisory Control and Data Acquisition}
\newacronym[longplural={Frames per Second}]{fpsLabel}{FPS}{Frame per Second} % provide the defined acronyms to be used
	%
	% for Backlinks to work properly
	%
	\subfile{./tex/Presentation.tex}
	%
	\begin{envModeNot}[\presentations]
		%
		% for Backlinks to work properly
		%
		\let\hypercontentsline=\contentsline
		\renewcommand{\contentsline}[4]{\hypertarget{toc.#4}{}\hypercontentsline{#1}{#2}{#3}{#4}}%
		%
		\sisetup{output-decimal-marker = {,}}
		\pagenumbering{gobble} % used to prevent the page numbering
		%
		\begin{envDebug}
			\layout
			\textrm{Serif: \rmdefault}\par
			\textsf{Sans-Serif: \sfdefault}\par
			\texttt{Teletype: \ttdefault}
		\end{envDebug}
		%
		\begin{envModeNot}[\paper]
		%
		% evtl. Sperrvermerkseite
		% nur in begründeten Ausnahmefällen verwenden
		% Aufgrund gesetzlicher Bestimmungen ist eine Sperre maximal für fünf Jahre möglich
		%
		\sperrvermerk{5}
		%
		\end{envModeNot}
		%
		\maketitle % creates the title page
		\hypersetup{pageanchor=true}
		%
		\begin{envModeNot}[\paper]
		%
		\pagenumbering{Roman} 
		%
		% Abstracts
		\subfile{./tex/Dedication.tex}
		%
		\subfile{./tex/Widmung.tex}
		%
		\subfile{./tex/Kurzreferat.tex}
		\subfile{./tex/Abstract.tex}
		%
		\subfile{./tex/Preface_Acknowledgement.tex}
		\subfile{./tex/Vorwort.tex}
		\newpage
		%
		\fhvlists
		\end{envModeNot}
		%
		% INSERT your .tex files
		%
		%\subfile{./tex/Hauptteil.tex}
		%\subfile{./tex/xxx.tex}
		\subfile{./tex/Examples.tex}
		%
		% END INSERT
		%
		\glossaryAndBibliography
		%
		% uncomment this if you like a short CV
		% \subfile{./tex/Lebenslauf.tex}
		%
		\begin{envModeNot}[\paper]
			\newpage
			\appendix
			\addAppendix{
			%
			% INSERT your .tex files
			%
			\subfile{./tex/Anhang.tex}
			%
			% END
			%
			}
			% Statuory Declaration
			\statuoryDeclaration
			%
		\end{envModeNot}
	\end{envModeNot}
	%
\end{document}
