\documentclass[../Paper/Paper.tex]{subfiles}
\begin{document}
%
% start arabic page numbering from here
%
\paragraph{\textit{Kurzfassung:}}\textit{%
	Die Kurzfassung soll einen Überblick über die Arbeit geben, sowie den „roten Faden“ und die wichtigsten Details für die/den Leser/in liefern. Sie muss informativ sein, unabhängig ob sie alleine oder zusammen mit der Arbeit gelesen wird. Eine gute Kurzfassung hat zwischen 100 und 200 Worte und fasst kurz und prägnant die Thematik, das Ziel der Arbeit, die verwendeten Methoden und (Kern-)Ergebnisse bzw. Erkenntnisse zusammen.\\
	%
%	Als erstes, um eine grundlegende Basis zu schaffen, werden die Substandards \citeauthor{ieee_802.1qbv_enhancements_2015} und \citeauthor{ieee_802.1qcc_stream_2018} beleuchtet und ebenfalls das \acrlong{bluecom} das die Grundlage der zwei Use-Cases darstellt, sowie \acrshort{snmp}, da dies Bestandteil von \citeauthor{ieee_802.1qcc_stream_2018} ist.Darauffolgend werden die zwei Use-Cases ausgearbeitet und anschließend ein Architektur Design für die Use-Cases und der hybriden Netzwerkkonfiguration von \acrshort{tsn} ausgearbeitet. Danach erfolgt die Implementierung des Designs und Verifikation der zwei Use-Cases.\\
	%
%	Als Ergebnis dieser Arbeit wird gezeigt, dass eine hybride \acrshort{tsn} Konfiguration mit dem Prototypen ermöglicht wird. Die Funktionalität wird mittels eines Langzeittests der zwei Use-Cases verifiziert.\\
	\textbf{\textit{Schlüsselwörter:}} Liste 4-5 Schlüsselwörter.
}\raggedbottom
%
\end{document}