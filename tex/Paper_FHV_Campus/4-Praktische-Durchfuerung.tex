\documentclass[./\jobname.tex]{subfiles}
\begin{document}
%
\newcounter{DataCnt}%
%
\section{PRAKTISCHE DURCHFÜHRUNG}\label{chap: Use-Case berscheibung}\raggedbottom%
%
Dieses Kapitel ist nur in der Variante \enquote{Arbeit mit Praxisanteil} enthalten.\par
%
Ausgehend von den Grundlagen werden die eigenen Ideen, Produkte, Konzepte, Technologien oder Software entwickelt. Dieser Prozess wird umfassend dokumentiert. Zum Abschluss werden die Ergebnisse in nachvollziehbarer Form dargestellt. Die Gliederung dieses Kapitels enthält in der Regel folgende Punkte: Planung (kann auch ähnlich wie in einem Projekthandbuch geschildert werden), gegebenenfalls Spezifikation, Umsetzung, Ergebnisse, eventuell Überprüfung der Ergebnisse.\par
%
Bei Arbeiten mit wirtschaftlichem Hintergrund oder bei stark unternehmensbezogenen Fragestellungen empfiehlt sich auch eine detaillierte Beschreibung des Unternehmens¬umfeldes. Bei Untersuchungen oder Messaufbauten sollten die verwendeten Verfahren oder Messinstrumente ebenfalls kurz geschildert werden.

%
\end{document}