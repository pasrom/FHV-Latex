\documentclass[./\jobname.tex]{subfiles}
\begin{document}
%\clearpage
\setcounter{page}{2}
\chapter*{Kurzreferat}
\section*{Deutscher TITEL}
%
Ein Kurzreferat macht die Relevanz der Arbeit sowie die innovativen Gedankengänge ersichtlich. Alleiniges Ziel ist es, in jeweils einem Absatz einen gerafften Überblick der Arbeit zu geben, so dass die Nutzer/innen entscheiden können, ob die vorliegende Arbeit für das eigene Forschungsvorhaben relevant ist oder nicht. Dementsprechend müssen darin die zentralen Abschnitte in neutraler, nicht wertender Perspektive beschrieben werden, vergleichbar einem Text über den Text von einem imaginierten Dritten.\par
%
Das Abstract muss für sich alleine verständlich sein. Es sollte zudem die zentralen Schlagwörter, die das Thema der Arbeit treffend umreißen, enthalten, um eine Indexierung in einer bibliographischen Referenzdatei zu erleichtern. Der Umfang von 1200 Anschlägen (d.h. Zeichen mit Leerzeichen; ca. 20 Zeilen) sollte nicht überschritten werden.\par
%
\blindtext[2]
\end{document}